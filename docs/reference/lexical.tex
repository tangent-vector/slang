\chapter{Lexical Structure}

\section{Source Units}

A \definition{source unit} of comprises a sequence of zero or more \definition{characters} which for the purposes of this section are defined as Unicode scalars (code points).
Implementations \may accept source units stored as files on disk, strings in memory, or any appropriate implementation-specified means.

\section{Encoding}

When encoding is required, source units units \should be encoded using UTF-8.
Implementations \may support additional implementation-specified encodings.

\section{Whitespace}

\definition{Horizontal whitespace} consists of space (U+0020) and horizontal tab (U+0009).

A \definition{line break} consists of a line feed (U+000A), carriage return (U+000D) or a carriage return followed by a line feed (U+000D, U+000A).
Line breaks are used as line separators rather than terminators; it not necessary for a source unit to end with a line break.

\section{Escaped Line Breaks}

An \definition{escaped line break} comprises a backslash (\lstinline|\\|, U+????) followed by a line break.

\section{Comments}

A \definition{comment} is either a line comment or a block comment.

A \definition{line comment} comprises two forward slashes (\lstinline{/}, U+????), followed by zero or more code points that do not contain a line break. A line comment extends up to, but not including, a subsequent line break or the end of the source unit.

A \definition{block comment} begins with a forward slash (\lstinline{/}, U+????) followed by an asterisk (\lstinline|*|, U+????). A block comment is terminated by the next instance of an asterisk followed by a forward slash (\lstinline|*/|).
A block comment contains all code points between where it begins and where it terminates, including any line breaks.
Block comments do not nest.
It is an error if a block comment is not terminated.

\section{Phases}

Compilation proceeds \emph{as if} the following steps are executed in order:

\begin{enumerate}
	\item Line numbering (for subsequent diagnostic messages) is noted based on the locations of line breaks
	\item Escaped line breaks are eliminated. No new code points are inserted to replace them. Any new escaped line breaks introduced by this step are not eliminated.
	\item All comments are replaced with a single space (U+0020).
	\item Preprocessing is done, resulting in a sequence of tokens
	\item Subsequent processing is performed on the sequence of tokens
\end{enumerate}

\section{Identifiers}

An \definition{identifier} begins with an uppercase or lowercase ASCII letter \lstinline|A| through \lstinline|Z|, or an underscore (\lstinline|_|).
After the first character, ASCII digits \lstinline|0| through \lstinline|9| are also allowed.

The identifier consisting of a single underscore (\lstinline|_|) is reserved by the implementation.

Otherwise, there are no fixed keywords or reserved words.
Words that name built-in language syntax can also be used as user-defined identifiers and will shadow the built-in definition in the scope of their definition.

\section{Literals}

\subsection{Integer Literals}

\subsection{Floating-Point Literals}

\subsection{String Literals}

\subsection{Character Literals}

\section{Operators and Punctuation}



