\chapter{Types}

\section{Terminology}

A \definition{scalar} type is one that represents a single value rather than a composite.

An \definition{numeric} type is one that can be used for traditional mathematical operations like addition, subtraction, etc.

An \definition{integer} type is a numeric type that can only take on integer values.
An \definition{unsigned integer} type can only take on non-negative values.
A \definition{signed integer} type uses two's complement notation to represent signed and unsigned values.

A \definition{floating-point} type is a numeric type that uses a sign bit, exponent, and mantissa to represent values that can include non-integer real values, positive/negative infinity, and also special not-a-number values.

\section{Layout}

The layout of a type when stored in memory may depend on the compilation target and how the type is used.
For example, different targets may lay out constant buffer memory differently from one another, and on a single target types in constant buffer memory may be laid out differently from types in structured buffers.

\section{Void Type}

The \lstinline|void| contains no data and has a single unnamed value.
The \lstinline|void| type is not a scalar type.

\section{Scalar Types}
\label{sec:scalar-types}

Scalar types must be atomically loadable and storable.
That is, loading a scalar type from memory must depend only on the bits in memory that define its value, and storing a scalar type to memory must not have an observable effect on other bits in memory.


\subsection{Boolean Type}

The \lstinline|bool| type represents a conditional value that can take on the values \lstinline|true| and \lstinline|false|.

\subsection{Scalar Integer Types}

The following scalar integer types are supported:

\begin{tabular}{lr}
\lstinline|int8_t|   &  8-bit signed integer   \\
\lstinline|int16_t|  & 16-bit signed integer   \\
\lstinline|int|      & 32-bit signed integer   \\
\lstinline|int64_t|  & 64-bit signed integer   \\
\lstinline|uint8_t|  &  8-bit unsigned integer \\
\lstinline|uint16_t| & 16-bit unsigned integer \\
\lstinline|uint|     & 32-bit unsigned integer \\
\lstinline|uint64_t| & 64-bit unsigned integer \\
\end{tabular}

\subsection{Scalar Floating-Point Types}

The following scalar floating-point types are supported:

\begin{tabular}{lr}
\lstinline|half| & 16-bit half-precision floating-point \\
\lstinline|float| & 32-bit single-precision floating-point \\
\lstinline|double| & 64-bit double-precision floating-point \\
\end{tabular}

\section{Vector Types}

A \definition{vector type} consists of a fixed number of \definition{components} of some scalar type.
A vector type is formed by \lstinline|vector<T,N>| where \lstinline|T| is a scalar type and \lstinline|N| is a constant integer.
The type \lstinline|T| is called the \definition{element type} or \definition{component type} of the vector, and \lstinline|N| is called the \definition{element count} or \definition{component count}.

A vector type is numeric, integer, signed integer, unsigned integer, and/or floating point if and only if its element type is.

The element count of a vector type must be greater than or equal to two, and less than or equal to 4.

For each of the scalar types defined in \label{sec:scalar-types}, there exist type aliases of the form \lstinline|<scalar type name><digit>| where \lstinline|<digit>| is one of \lstinline|2|, \lstinline|3|, or \lstinline|4|.

\section{Matrix Types}

A \definition{matrix type} is formed by \lstinline|matrix<T,R,C>| where \lstinline|T| is a scalar type and \lstinline|R| and \lstinline|C| are constant integers.
The matrix type \lstinline|matrix<T,R,C>| consists of \lstinline|R| \definition{rows} of type \lstinline|vector<T,C>| (the \definition{row type} of the matrix type).
The type \lstinline|T| is called the \definition{element type} of the matrix, \lstinline|R| is called the \definition{row count}, and \lstinline|C| is called the column count.

The row count and column count of a matrix type must be greater than or equal to two, and less than or equal to 4.

For each of the scalar types defined in \label{sec:scalar-types}, there exist type aliases of the form \lstinline|<scalar type name><row-digit>x<col-digit>| where \lstinline|<row-digit>| and \lstinline|<col-digit>| are each one of \lstinline|2|, \lstinline|3|, or \lstinline|4|.

\section{Structure Types}

A \definition{structure type} is introduced using a structure declaration (Section~\ref{sec:struct-declaration}).
For example:

\begin{lstlisting}
struct Sphere
{
    float3 position;
    float  radius;
}
\end{lstlisting}

In this example, \lstinline|Sphere| is the name of a user-defined structure type.
Variables declared in a struct declaration (in this case, \lstinline|position| and \lstinline|radius|) are \definition{member variables} (also: \definition{fields}).
A structure type may contain zero or more member variables.

\section{Enumeration Types}

An \definition{enumeration type} is introduced using an enumeration declaration (Section~\ref{sec:enum-declaration}).
For example:

\begin{lstlisting}
enum ColorCompoment
{
    Red,
    Green,
    Blue,
}
\end{lstlisting}

In this example, \lstinline|ColorComponent| is the name of a user-defined enumeration type.
An enumeration type may contain zero or more \definition{cases}.
In this example, \lstinline|Red|, \lstinline|Green|, and \lstinline|Blue| are the cases of the type \lstinline|ColorComponent|).

\section{Arrays}

An \definition{array type} is either a sized array type or an unsized array type.

A \definition{sized array type} can be written \lstinline|T[N]| where \lstinline|T| is a type and \lstinline|N| is a constant integer.
The type \lstinline|T| is the \definition{element type} of the sized array type, and \lstinline|N| is the \definition{element count}.

The element count of a sized array type must be greater than or equal to zero.

An \definition{sized array type} can be written \lstinline|T[]| where \lstinline|T| is a type.
The type \lstinline|T| is the \definition{element type} of the unsized array type.

\section{Opaque Types}

The Slang library defines a number of \definition{opaque types} such as \lstinline|Texture2D|.

\section{Type Aliases}

A \definition{type alias} is introduced with a \lstinline|typedef| declaration.
For example:

\begin{lstlisting}
typedef float RadianAngle;
\end{lstlisting}

In this example, the identifier \lstinline|RadianAngle| has been declared to be an alias for \lstinline|float|.

A type alias is semantically equivalent to the type it is an alias of.
Implementations may elect to use type aliases in diagnostic messages if they can improve clarity.

\section{Generic Type Parameters}

\section{Associated Types}

